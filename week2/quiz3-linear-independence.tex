\documentclass[12pt]{article}
\usepackage{amsmath} % for 'bmatrix' environmen
\usepackage{commath}
\usepackage{graphicx}
\usepackage[utf8]{inputenc}
\usepackage[english]{babel}
\usepackage[document]{ragged2e}
\begin{document}
    \section*{Question 1}
    \begin{flushleft}
        In the lecture videos you saw that vectors are linearly dependent if it is possible
        to write one vector as a linear combination of the others. For example, the vectors
        a, b and c are linearly dependent if a =  $q_1b + q_2c$ where $q_1$ and $q_2$ are scalars. \\
    \end{flushleft}
    \begin{flushleft}
        Are the following vectors linearly dependent? \\
        $
            a =
            \begin{bmatrix}
                1 \\ 1
            \end{bmatrix}
            \text{ and } b = 
            \begin{bmatrix}
                2 \\ 2
            \end{bmatrix}
        $
    \end{flushleft}
    \subsection*{Answer}
    \begin{flushleft}
        Yes. Vectors are linearly independent if
        $
            b_3 \neq a_1b_1 + a_2b_2
        $ \\
        In other words, if some scalar exists to make the vectors equivalent, they are not linearly independent. \\
        In our case:
        $
            a = \frac{1}{2}b
        $
         which means the vectors are linearly dependent and NOT linearly independent.
    \end{flushleft}
    \section*{Question 2}
    \begin{flushleft}
        Are the following vectors linearly independent? \\
        $
            a = 
            \begin{bmatrix}
                1 \\ 0 \\ 1
            \end{bmatrix}
            b = 
            \begin{bmatrix}
                2 \\ - 1 \\ 1
            \end{bmatrix}
            c = 
            \begin{bmatrix}
                -3 \\ 1 \\ -2
            \end{bmatrix}
        $
    \end{flushleft}
    \subsection*{Answer}
    \begin{flushleft}
        No. $ a = -b - c $ which makes these vectors linearly dependent.
    \end{flushleft}
\end{document}